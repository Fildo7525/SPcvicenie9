\documentclass{article}

\usepackage{graphicx}
\usepackage[hidelinks]{hyperref}
\usepackage[a4paper, total={6in, 8in}]{geometry}
\usepackage[slovak]{babel}
\usepackage{caption}
\usepackage{subcaption}

\graphicspath{./include/}

\renewcommand{\figurename}{Obr.}
\renewcommand{\contentsname}{Obsah}

\begin{document}

\begin{titlepage}
	\null\vfill

	\begin{center}
		{\Huge Dvojpolohová regulácia teploty tepelného systému }
		\vskip 2cm

		{\Large Cvičenie č. 9}
		\vskip 0.5cm

		{\large Spojité procesy}
	\end{center}

	\vfill
	\vfill

	\begin{flushright}
		Filip Lobpreis \\
		Matúš Machata \\
		\small\today\\
	\end{flushright}
	\hfill
\end{titlepage}

\thispagestyle{empty}
\clearpage

\tableofcontents
\thispagestyle{empty}
\clearpage

\section{Zadanie}
\label{sec:zadanie}
\pagenumbering{arabic}

\begin{figure}[!htbp]
	\begin{center}
		\label{fig:zadanie}
		\includegraphics[width=0.8\textwidth]{./include/zadanie.png}
	\end{center}
	\caption{Zadanie z cvičenia č. 9 z predmetu spojité procesy}
\end{figure}

\clearpage

\section{Pomôcky}
\label{sec:pomocky}

V tomto zadaní je našou úlohou od simulovať návrh nespojitej dvojpolohovej regulácie teploty na laboratórnom
modeli tepelného systému a overiť navrhované riešenie. Pri tomto zadaní sme použili už preddefinovanú schému
v programe \textit{Simulink} (Obr.~\ref{fig:schema}).

\begin{figure}[!htbp]
	\begin{center}
		\includegraphics[width=0.95\textwidth]{./include/schema.png}
		\label{fig:schema}
		\caption{Simulacna schema reprezentujuca dvojpolohovu regulaciu tepelneho systemu.}
	\end{center}
\end{figure}

V tomto zapojení vidíme tri vstupné signály, tie sú už preddefinované. \textbf{Tref} (referenčná teplota)
má nasledujúci priebeh (viď Obr.~\ref{fig:ziadanaHodnota}).

\begin{figure}[!htbp]
	\begin{subfigure}{0.5\textwidth}
		\includegraphics[width=\textwidth]{./include/ziadanaHodnota.png}
		\label{fig:ziadanaHodnota}
		\caption{Priebeh referenčnej teploty v priebehu 1800 sekúnd.}
	\end{subfigure}
	\hfill
	\begin{subfigure}{0.5\textwidth}
		\includegraphics[width=\textwidth]{./include/porucha.png}
		\label{fig:porucha}
		\caption{Priebeh poruchy systému v priebehu 1800 sekúnd.}
	\end{subfigure}
\end{figure}

Signál \textbf{d} reprezentuje poruchu systému. Poruchovou veličinou sú otáčky ventilátora, ktoré sú závislé
od napätia na ventilátore. Toto napätie sa pohybuje v rozsahu od 20\% do 100\% (Obr.~\ref{fig:porucha}).

\section{Merania}
\label{sec:merania}

\subsection{Hysteréza s hodnotou 0,2}
\label{sec:meranie1}

V prvom zadaní sme zvolili hodnotu hysterézy 0,2. Výsledok simulácie môžeme vidieť na obrázkoch
Obr.~\ref{fig:m1t2} a Obr.~\ref{fig:m1u}.

\begin{figure}[!htbp]
	\begin{center}
		\includegraphics[width=\textwidth]{./include/m1T2.png}
	\end{center}
	\caption{Graf žiadanej a meranej hodnoty teploty na snímači T2 v prvom meraní [°C].}
	\label{fig:m1t2}
\end{figure}

\clearpage

\begin{figure}[!htbp]
	\begin{center}
		\includegraphics[width=\textwidth]{./include/m1u.png}
	\end{center}
	\caption{Graf výkonu výhrevnej špirály v prvom meraní [\%].}
	\label{fig:m1u}
\end{figure}

V prvej časti merania (od času 0s po 300s) máme žiadanú teplotu 28°C s ventilátorom zapnutým na 100\%.
V druhej časti merania (od času 300s po 600s) máme žiadanú teplotu 31°C s ventilátorom zapnutým na 100\%.
V tretej časti merania (od času 600s po 900s) máme žiadanú teplotu 31°C s ventilátorom zapnutým na 25\% a
v poslednej časti merania (od času 900s po 1800s) má systém vertikálne investované správanie ako prvé
dve časti merania. Systém ma žiadanú hodnotu teploty nastavenú na 31°C a ventilátor je zapnutý na 100\%
a prechádza do stavu, kde je žiadaná hodnota teploty 28°C a ventilátor zostane zapnutý na 100\%.
Priebeh teploty na snímači \textit{T2} vidíme na Obr.~\ref{fig:m1t2} a priebeh výkonu výhrevnej špirály
môžeme vidieť na obrázku Obr.~\ref{fig:m1u}. Spôsob správania výhrevnej špirály je opísaný v zadaní.
\ref{sec:zadanie}.

\clearpage

\subsection{Hysteréza s hodnotou 0,5}
\label{sec:meranie2}

\begin{figure}[!htbp]
	\begin{center}
		\includegraphics[width=\textwidth]{./include/m2T2.png}
	\end{center}
	\caption{Graf žiadanej a meranej hodnoty teploty na snímači T2 v druhom meraní [°C].}
	\label{fig:m2t2}
\end{figure}

\clearpage

\begin{figure}[!htbp]
	\begin{center}
		\includegraphics[width=\textwidth]{./include/m2u.png}
	\end{center}
	\caption{Graf výkonu výhrevnej špirály v druhom meraní [\%].}
	\label{fig:m2u}
\end{figure}

V druhom meraní máme zvolenú hodnotu hysterézy 0,5. Rozdelenie časti simulácie je rovnaké ako v prvom meraní.
Rozdiel nastáva v responzívnosti systému na zmenu teploty. V druhom meraní je responzívnosť systému menšia
kvôli už spomínanej vyššej hodnote hysterézy. Ako môžeme vidieť na obrázku Obr.~\ref{fig:m2t2}, frekvencia
meranej teploty na snímači T2 je menšia a zároveň aj strmosť tejto zmeny je menšia.

\subsection{Hysteréza s hodnotou 0,8}
\label{sec:meranie3}

\begin{figure}[!htbp]
	\begin{center}
		\includegraphics[width=\textwidth]{./include/m3T2.png}
	\end{center}
	\caption{Graf žiadanej a meranej hodnoty teploty na snímači T2 v treťom meraní [°C].}
	\label{fig:m3t2}
\end{figure}

\clearpage

\begin{figure}[!htbp]
	\begin{center}
		\includegraphics[width=\textwidth]{./include/m3u.png}
	\end{center}
	\caption{Graf výkonu výhrevnej špirály v treťom meraní [\%].}
	\label{fig:m3u}
\end{figure}

Tak isto ako pri druhom meraní, frekvencia a strmosť zmeny teploty T2 sa zmenšili kvôli zvýšeniu hodnoty hysterézy čo môžeme vidieť na Obr.~\ref{fig:m3t2}. Preto sa aj perióda zapínania výhrevnej špirály zmenšila ako môžeme vidieť na Obr.~\ref{fig:m3u}.

\section{Zhrnutie}
\label{sec:zhrnutie}

\end{document}

